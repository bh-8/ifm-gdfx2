\documentclass{article}

\usepackage[german]{babel}
\usepackage{geometry}
\usepackage{hyperref}

\title{Generalizable Deepfake Detection Framework}
\author{Bernhard Birnbaum}

\begin{document}
    \maketitle

    \section{Motivation \& Stand der Technik}
    \begin{itemize}
        \item Basics~\footnote{Machine Learning | Google for Developers, \url{https://developers.google.com/machine-learning}}~\footnote{Dive into Deep Learning, \url{https://d2l.ai/}}
        \item Deep Learning Book~\cite{deeplearningbook}
    \end{itemize}
    \begin{itemize}
        \item Entwicklungsumgebung/Cluster: MiniForge
    \end{itemize}
    \subsection{PyTorch vs. Tensorflow}
    \begin{itemize}
        \item Vergleich und Abwägung
    \end{itemize}
    \subsection{Resnet}
    \subsection{BiLSTMs}

    \section{Konzept}
    \subsection{DF40-Datenset}
    \begin{itemize}
        \item \cite{yan2024df40}
        \item Ordnerstruktur der Daten
    \end{itemize}
    \subsection{Splits}
    \begin{itemize}
        \item Anzahl Elemente Training/Test pro Klasse
    \end{itemize}
    \subsection{Model}
    \begin{itemize}
        \item BiLSTM/Layer
    \end{itemize}
    \subsection{Training}
    \subsection{Validation}

    \section{Implementierung}
    \begin{itemize}
        \item 
    \end{itemize}

    \section{Evaluation}
    \begin{itemize}
        \item Model Performance anhang von Metriken
    \end{itemize}

    \section{Zusammenfassung \& Fazit}

    \bibliographystyle{plain}
    \bibliography{refs}
\end{document}
