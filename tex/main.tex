\documentclass{article}

\usepackage[german]{babel}
\usepackage{geometry}
\usepackage{hyperref}

\title{Generalizable Deepfake Detection Framework}
\author{Bernhard Birnbaum}

\begin{document}
    \maketitle

    \section{Motivation \& Stand der Technik}
    \begin{itemize}
        \item Basics~\footnote{Machine Learning | Google for Developers, \url{https://developers.google.com/machine-learning}}~\footnote{Dive into Deep Learning, \url{https://d2l.ai/}}
        \item Deep Learning Book~\cite{deeplearningbook}
    \end{itemize}
    \subsection{Resnet}
    % https://medium.com/@nitishkundu1993/exploring-resnet50-an-in-depth-look-at-the-model-architecture-and-code-implementation-d8d8fa67e46f
    \subsection{BiLSTMs}
    \begin{itemize}
        \item RNNs
        \item LSTMs
        \item BiLSTMs % https://medium.com/@anishnama20/understanding-bidirectional-lstm-for-sequential-data-processing-b83d6283befc
    \end{itemize}
    \section{Konzept}
    \subsection{DF40-Datenset}
    \begin{itemize}
        \item \cite{yan2024df40}
        \item Ordnerstruktur der Daten
        \item Preprocessing/Normalisierung/Mischen
    \end{itemize}
    \subsection{Splits}
    \begin{itemize}
        \item Anzahl Elemente Training/Test pro Klasse
    \end{itemize}
    \subsection{Model}
    \begin{itemize}
        \item BiLSTM/Layer
        \item DropOut-Layer
        \item L2-Regularisierung
    \end{itemize}
    \subsection{Training}
    \begin{itemize}
        \item LR-Scheduler (ReduceLROnPlateau)
        \item Early-Stopping
    \end{itemize}
    \subsection{Validation}
    \begin{itemize}
        \item Metriken auc, categorical\_accuracy, f1\_score, precision, recall
    \end{itemize}
    \section{Implementierung}
    \begin{itemize}
        \item Entwicklungsumgebung/Cluster: MiniForge
        \item Python- und Tool-Versionen, Bibliotheken, ...
    \end{itemize}
    \subsection{PyTorch vs. Tensorflow}
    \begin{itemize}
        \item Vergleich und Abwägung
    \end{itemize}
    \section{Evaluation}
    \begin{itemize}
        \item Model Performance anhang von Metriken
    \end{itemize}

    \section{Zusammenfassung \& Fazit}

    \bibliographystyle{plain}
    \bibliography{refs}
\end{document}
